\section{Motivation of Quantum Algorithms for Time Simulation of Quantum Systems}
  The heart of simulation of physical systems is solving Schrödinger's equation of motion:

  \begin{equation}
    \mathrm{i}
    \label{eq:SchEqn}\pdv{\ket{\psi}}{t} = \hat{H} \ket{\psi}
  \end{equation}

  Where $\hat{H}$ is the Hamiltonian that defines the interaction between the system's components. In position representation, A one-dimensional system may be simulated by solving the equation

  \begin{equation}
    \mathrm{i}\pdv{\ket{\psi}}{t} = \Bigg[\sum_{i = 1}^{n} \frac{\hat{P}_i^2}{2m_i} + \hat{V}(x_1, x_2, \ldots, x_n)\Bigg] \ket{\psi}
  \end{equation}

  Supposed $\ket{\psi}$ represents an $n$-particle system state. Hence, a single particle dynamics may be simulated by solving the equation

  \begin{equation}
    \mathrm{i}\pdv{\ket{\psi}}{t} = \Bigg[\frac{\hat{P}^2}{2m} + \hat{V}(x)\Bigg] \ket{\psi}
  \end{equation}

  A classical algorithm may use a fine discretization of position basis, in some spatial region $\mathit{S} = [0,L]$, with a basis of $N$ statevectors and a discretization step $\Delta x = L/(N-1)$. Such that

  \begin{equation}
    \ket{x} \text{ for } x \in S \rightarrow \ket{k \Delta x} \text{ for } k = 0,1,\ldots,N-1
  \end{equation}

  This scheme leads to a representation of any single particle position state as a linear combination of discretized statevectors

  \begin{equation}
    \ket{\psi} = \sum_{k = 0}^{N-1} a_k \ket{k \Delta x}
  \end{equation}

\section{Section 2}
  \lipsum[2-4]

\section{Section 3}
  \lipsum[2-4]

\section{Section 4}
  \lipsum[2-4]

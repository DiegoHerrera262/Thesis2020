An introduction to quantum time simulation, as opposed to \textit{classical} time simulation of quantum systems is presented on \autoref{sec:qtsVcts}. After that, a generic framework for approximate quantum time evolution is presented on \autoref{sec:trotter} Finally, on \autoref{sec:hubbard}, quantum digital simulations of one-dimensional Hubbard models, carried out by Las Heras et. al. \cite{HubbardSimulLasHeras} and Barends et. al. \cite{HubbardSimul}, are discussed as immediate predecessors of this works.

\section{Quantum Time Simulation v. Classical Time Simulation}
\label{sec:qtsVcts}

  At the heart of simulation of quantum physical systems is solving Schrödinger's equation of motion \cite{Beck, Nielsen}:

  \begin{equation}
  \mathrm{i}
  \label{eq:SchEqn}\pdv{\ket{\psi}}{t} = \hat{H} \ket{\psi}
  \end{equation}

  Where $\hat{H}$ is the Hamiltonian that defines the interaction between the system's components, and perhaps its environment. In position representation, A one-dimensional system of spinless particles can be simulated by solving the equation

  \begin{equation}
  \mathrm{i}\pdv{\ket{\psi}}{t} = \Bigg[\sum_{i = 1}^{n} \frac{\hat{P}_i^2}{2m_i} + \hat{V}(x_1, x_2, \ldots, x_n)\Bigg] \ket{\psi}
  \end{equation}

  Supposed $\ket{\psi}$ represents an $n$-particle system state. Hence, a single particle dynamics can be determined by solving the equation

  \begin{equation}
  \mathrm{i}\pdv{\ket{\psi}}{t} = \Bigg[\frac{\hat{P}^2}{2m} + \hat{V}(x)\Bigg] \ket{\psi}
  \end{equation}

  A classical algorithm may use a fine discretization of position basis, in some spatial region $\mathit{S} = [0,L]$, with a basis of $N$ statevectors and a discretization step $\Delta x = L/(N-1)$. Such that

  \begin{equation}
    \ket{x} \text{ for } x \in S \rightarrow \ket{k \Delta x} \text{ for } k = 0,1,\ldots,N-1
  \end{equation}

  This scheme leads to a representation of any single particle position state as a linear combination of discrete statevectors

  \begin{equation}
    \ket{\psi(t)} = \sum_{k = 0}^{N-1} a_k(t) \ket{k \Delta x}
  \end{equation}

  Momentum operator could be approximated using finite difference formulas, thus leading to a system of coupled differential equation on the expansion coefficients $a_k$

  \begin{equation}
    \mathrm{i}\pdv{a_k}{t} = \sum_{l = 0}^{N} H_{kl} a_l
    \label{eq:MatrixSchr}
  \end{equation}

  Solution of equation \ref{eq:MatrixSchr}, given $a_k(0)$, would yield a complete knowledge of the particle's dynamics at any time. Typically, this would require diagonalization of the Hamiltonian matrix, $H_{kl}$. There are more efficient approaches than this, of course. For instance, Numerov integration. However, the goal of this example is to introduce quantum time simulation on a digital quantum computer as smoothly as possible.

  Time simulation on a digital quantum computer could be carried out in a very different way \cite{Strini, Nielsen}. A schematic is presented on figure \ref{fig:timevolslice}. Pretty much in the same way as in the naïve example discussed before, position basis may be discretized. Nevertheless, this time the coefficients would be encoded directly as the amplitudes of the statevector of an $n$-qubit computer. It is readily seen that to achieve a discretization with $N$ system statevectors, only $\mathcal{O}(\log{N})$ qubits are needed. Although not very significant for a single particle, this illustrates that quantum computers have inherent exponential advantage over classical computers in terms of space resources. Instead of diagonalizing the Hamiltonian, digital quantum time simulation relies on the direct solution

  \begin{equation}
    \ket{\psi(t)} = \exp(-\mathrm{i}\int_{t_0}^t \hat{H}(t)dt)\ket{\psi(0)}
    \label{eq:UnitaryEvolution}
  \end{equation}

  Which for time-independent Hamiltonians reduces to

  \begin{equation}
    \ket{\psi(t)} = \mathrm{e}^{-\mathrm{i}(t - t_0) \hat{H}}\ket{\psi(0)}
    \label{eq:UnitaryEvolutionNoTime}
  \end{equation}

  In the naïve example considered until now, time evolution with a digital quantum computer amounts to computing unitary operator ($t_0 = 0$)

  \begin{equation}
    \hat{U}(t) = \mathrm{e}^{-\mathrm{i}t \frac{\hat{P}^2}{2m}}\mathrm{e}^{-\mathrm{i}t \hat{V}(x)} + \mathcal{O}(t^2)
  \end{equation}

  Using Baker-Haussdorf formulae, or a Suzuki-Trotter scheme of higher order, better expressions for the time evolution operator of the system may be obtained. Notice that operator

  \[
  \hat{U}_P = \mathrm{e}^{-\mathrm{i}t \frac{\hat{P}^2}{2m}}
  \]

  Is efficiently computable, using the Quantum Fourier Transform Algorithm:

  \[
  \hat{U}_P = QFT\mathrm{e}^{-\mathrm{i}t \frac{\hat{x}^2}{2m}}QFT^{\dagger}
  \]

  Therefore, if operator

  \[
  \hat{U}_V = \mathrm{e}^{-\mathrm{i}t \hat{V}(x)}
  \]

  Is efficiently computable, time simulation on a quantum computer might be more resource-friendly, both in terms of space and time, than common simulation using classical computers. Only theoretical constraints are error bounds for a given simulation time interval. Given simulation time and error bound, a time slice $dt$ is fixed, and thus repeated application of operator $\hat{U}(dt)$ evolves a single particle state from some initial state $\ket{\psi(0)}$, to state $\ket{\psi(t)}$.

  \begin{figure}
    \centering
    \begin{quantikz}
      \lstick[wires=4]{$\ket{\psi(0)}$} & \gate[wires=4]{QFT^{\dagger}} & \gate[wires=4]{\mathrm{e}^{-\mathrm{i}t \frac{\hat{x}^2}{2m}}} & \gate[wires=4]{QFT^{\dagger}} & \gate[wires=4]{\mathrm{e}^{-\mathrm{i}t \frac{\hat{V}(x)}{2m}}} & \qw  \rstick[wires=4]{$\ket{\psi(t)}$} \\
                                        &                               &                                                              &                               &                                                                 & \qw                                    \\
                                        &                               &                                                              &                               &                                                                 & \qw                                    \\
                                        &                               &                                                              &                               &                                                                 & \qw                                    \\
    \end{quantikz}
    \caption{Time evolution slice for a single particle state on a digital quantum computer. This step should be repeated several times, with a time step $dt$, which depends upon the desired error bound and spatial discretization.}
    \label{fig:timevolslice}
  \end{figure}

  In summary, rather than using physical bits to encode the expansion coefficients of a particle's state, like on a classical computer, a quantum algorithm relies on the nature of qubits to encode directly such a state. This leads to an exponential reduction in the space complexity of the problem. Furthermore, multi-qubit gates can be used to implement unitary evolution, without the explicit need of matrix diagonalization. Thus leading to a potentially faster evolution simulation. For a simple system like this, codification of the information of all expansion coefficients would require at least $2N$ reals parameters. Furthermore, diagonalization of the Hamiltonian matrix, $H_{kl}$, would require $\mathcal{O}(N^2)$ computational steps. As a result, the simulation advantage posed by quantum computation seems unnecessary. Also, there are numerous efficient classical algorithms for solving Schrodinger's time dependent equation, such as Numerov or Runge-Kutta integration. However, this example illustrates the difference between classical simulation and quantum simulation using a digital quantum computer, and some of the possible advantages of quantum time simulation using the former type of information processors.

  For a one-dimensional system of several particles, equation \ref{eq:MatrixSchr} can be generalized easily. However, the number of coefficients required to describe a statevector in a discrete basis would grow exponentially with the number of particles. As well as matrix Hamiltonian size. As a result, simulation of time dynamics on a classical computer results impractical. Therefore, quantum time simulation of multi-particle systems is an application to which digital quantum computers may represent a practical advantage. In the following sections, common techniques for quantum time simulation in digital computers are presented. In particular, simple approximation formulas are discussed and compared. 

  %% Here I would have to talk more about...

\section{Common Approximation Schemes for Unitary Evolution}
\label{sec:trotter}

  Consider a system of $N$ components, whose Hamiltonian can be expressed as a sum of local Hamiltonians (i. e. that model interaction between at most $C$ components) \cite{Nielsen,LloydNature}

  \begin{equation}
    \hat{H} = \sum_{k = 1}^{L} \hat{H}_k
    \label{eq:SparseHam}
  \end{equation}

  Where $L$ is some polynomial on the number of system components. In general, $[\hat{H}_i,\hat{H}_j] \neq 0$, and thus

  \begin{equation}
    \mathrm{e}^{-\mathrm{i}\hat{H}t} \neq \prod_{k = 1}^{L} \mathrm{e}^{-\mathrm{i}\hat{H}_kt}
    \label{eq:CommuteUnit}
  \end{equation}

  Many systems are described by local interactions, for instance, electrons in a solid material or magnetic moments in a lattice. In several cases, local interaction Hamiltonians are non-commuting, and thus approximation methods are necessary for performing time evolution. In this section, schemes for approximating unitary evolution of a quantum system are discussed.

  \subsection{Trotter Formulas}

  Consider operators $\hat{H}_1$, $\hat{H}_2$, with $[\hat{H}_1,\hat{H}_2] \neq 0$. By definition

  \begin{align}
    \mathrm{e}^{-\mathrm{i}\hat{H}_1 t} & = \sum_{m = 0}^{\infty} \frac{(-\mathrm{i}t)^m}{m!}\hat{H}_1^m \\
    \mathrm{e}^{-\mathrm{i}\hat{H}_2 t} & = \sum_{l = 0}^{\infty} \frac{(-\mathrm{i}t)^l}{l!}\hat{H}_2^l
    \label{eq:ExpSeries}
  \end{align}

  It is readily shown that

  \begin{equation}
    \mathrm{e}^{-\mathrm{i}\hat{H}_1 t}\mathrm{e}^{-\mathrm{i}\hat{H}_2 t} = \sum_{k = 0}^{\infty} \frac{(-\mathrm{i}t)^k}{k!} \Bigg[\sum_{m = 0}^k \binom{k}{m} \hat{H}_1^m \hat{H}_2^{k-m}\Bigg]
    \label{eq:ExpProdExact}
  \end{equation}

  Fon non-commuting operators, it is so that

  \begin{equation}
    \sum_{m = 0}^k \binom{k}{m} \hat{H}_1^m \hat{H}_2^{k-m} = (\hat{H}_1 + \hat{H}_2)^k + f_k(\hat{H}_1,\hat{H}_2)
    \label{eq:BinomialTheorem}
  \end{equation}

  Where $f_k(\hat{H}_1,\hat{H}_2)$ is a function of the commutator of the operators. Since $f_1(\hat{H}_1,\hat{H}_2) = 0$, it is so that

  \begin{equation}
    \mathrm{e}^{-\mathrm{i}\hat{H}_1 t}\mathrm{e}^{-\mathrm{i}\hat{H}_2 t} = \mathrm{e}^{-\mathrm{i}(\hat{H}_1 + \hat{H}_2) t} + \mathcal{O}(t^2)
    \label{eq:O2Approx}
  \end{equation}

  If $|t| \ll 1$, the product of the exponential operators estimate the evolution operator with an error $\mathcal{O}(t^2)$. In the general case, it must be noted that

  \begin{equation}
    \mathrm{e}^{-\mathrm{i}\sum_{k = 0}^L \hat{H}_k t} = \hat{1} + (-\mathrm{i}t)\sum_{k = 0}^L \hat{H}_k + \frac{(-\mathrm{i}t)^2}{2} \Bigg[\sum_{k = 0}^L \hat{H}_k^2 + 2 \sum_{j > k}\hat{H}_k \hat{H}_j\Bigg] + \mathcal{O}(t^3)
    \label{eq:TrotterFormula}
  \end{equation}

  In consequence, a unitary evolution operator with local interactions may be approximated, to quadratic order, by the exponential product

  \begin{equation}
    \mathrm{e}^{-\mathrm{i}\sum_{k = 0}^L \hat{H}_k t} = \prod_{k = 1}^{L} \mathrm{e}^{-\mathrm{i}\hat{H}_kt} + \mathcal{O}(t^2)
    \label{eq:2ndOrderTrotter}
  \end{equation}

  In some instances, quadratic approximations may be enough. In his seminal paper, Lloyd presents this quadratic approximation for simulation of quantum systems with local interaction \cite{LloydNature}. Also, Las Heras et. al. simulate a Hubbard Hamiltonian with up to 4 fermionic modes using second order approximations to unitary evolution \cite{HubbardSimul, HubbardSimulLasHeras}. However, in following sections, higher-order approximation schemes are discussed, based upon equation \ref{eq:TrotterFormula}.

  \subsection{Some Cubic Order Schemes}

  The first cubic order approximation discussed is the so called Baker-Haussdorf formulae \cite{Nielsen}. By series expansion, it can be shown that

  \begin{align*}
    \mathrm{e}^{-\mathrm{i}\hat{H}_1t}\mathrm{e}^{-\mathrm{i}\hat{H}_2t}\mathrm{e}^{-\mathrm{i}[\hat{H}_1,\hat{H}_2]t^2} = & \hat{1} + (-\mathrm{i}t) (\hat{H}_1 + \hat{H}_2) \\
    & + \frac{(-\mathrm{i}t)^2}{2}(\hat{H}_1^2 + \hat{H}_2^2 + \hat{H}_1\hat{H}_2 + \hat{H}_2\hat{H}_1) + \mathcal{O}(t^3) \\
    = & \mathrm{e}^{-\mathrm{i}(\hat{H}_1 + \hat{H}_1)t} + \mathcal{O}(t^3)
    \label{eq:Hausdorf1}
  \end{align*}

  Although useful in case of operators that constitute a Lie algebra, the formulae above may not be enough in other instances. A more general approximation formulae is

  \begin{equation}
    \mathrm{e}^{-\mathrm{i}t\sum_{l = 0}^{L-1}\hat{H}_l} = \Bigg(\prod_{l = 0}^{L-1}\mathrm{e}^{-\mathrm{i}\hat{H}_l\frac{t}{2}}\Bigg)\Bigg(\prod_{l = L-1}^{0}\mathrm{e}^{-\mathrm{i}\hat{H}_l\frac{t}{2}}\Bigg) + \mathcal{O}(t^3)
    \label{eq:Suzuki0}
  \end{equation}

  This can be deduced directly from the identity

  \begin{equation}
    \mathrm{e}^{-\mathrm{i}\sum_{l = 0}^{2L-1} \hat{K}_l t} = \hat{1} + (-\mathrm{i}t)\sum_{l = 0}^{2L-1} \hat{K}_l + \frac{(-\mathrm{i}t)^2}{2} \Bigg[\sum_{l = 0}^{2L-1} \hat{K}_l^2 + 2 \sum_{j > l}\hat{K}_l \hat{K}_j\Bigg] + \mathcal{O}(t^3)
    \label{eq:TrotterFormula1}
  \end{equation}

  With the identifications

  \begin{equation}
    \hat{K}_l = \hat{K}_{2L-1-l} = \frac{\hat{H}_l}{2}
    \label{eq:Identifications}
  \end{equation}

  And the following observations

  \begin{equation}
    \begin{gathered}
      \sum_{l = 0}^{2L-1} \hat{K}_l^2 = \frac{1}{2}\sum_{l = 0}^{L-1} \hat{H}_l^2 \\
      2\sum_{l' > l} \hat{K}_{l} \hat{K}_{l'} = \frac{1}{2}\sum_{l = 0}^{L-1} \hat{H}_l^2 + \sum_{l'> l} \Bigg( \hat{H}_{l} \hat{H}_{l'} + \hat{H}_{l'} \hat{H}_{l}\Bigg)\\
      \Bigg(\sum_{l = 0}^{L-1} \hat{H}_l \Bigg)^2 = \sum_{l = 0}^{L-1} \hat{H}_l^2 + \sum_{l'> l} \Bigg( \hat{H}_{l} \hat{H}_{l'} + \hat{H}_{l'} \hat{H}_{l}\Bigg)\\
    \end{gathered}
  \end{equation}

  \subsection{Suzuki - Trotter Scheme}
  This scheme is a higher order approximation to an evolution operator, that works iteratively on top of the approximation given by equation \ref{eq:Suzuki0}. Define \cite{SuzukiFormula, BerryErrorBounds}

  \begin{equation}
    \hat{S}_2(\lambda) = \prod_{l = 0}^{L-1} \mathrm{e}^{\frac{\lambda}{2}\hat{H}_l} \prod_{l = L-1}^{0} \mathrm{e}^{\frac{\lambda}{2}\hat{H}_l}
    \label{eq:SuzukiInit}
  \end{equation}

  Then, for $k > 1$, the following recursion relation is defined

  \begin{equation}
    \hat{S}_{2k}(\lambda) = [\hat{S}_{2k-2}(p_k\lambda)]^2 \hat{S}_{2k-2}((1 - 4 p_k)\lambda) [\hat{S}_{2k-2}(p_k\lambda)]^2
    \label{eq:SuzukiRecursion}
  \end{equation}

  Where coefficients $p_k$ are defined as

  \begin{equation}
    p_k = \frac{1}{4 - 4^{\frac{1}{2k-1}}}
    \label{eq:pk}
  \end{equation}

  It has been shown by Suzuki that \cite{SuzukiFormula}

  \begin{equation}
    \mathrm{e}^{-\mathrm{i}t \sum_{l = 0}^{L-1} \hat{H}_l} = \hat{S}_{2k}(-\mathrm{i}t) + \mathcal{O}(t^{2k+1})
  \end{equation}

  On \cite{BerryErrorBounds}, Barry et. al. show that Suzuki-Trotter schemes can efficiently simulate sparse Hamiltonians, such as the ones considered in this work. As a matter of fact, they have shown that the number of exponentials ($N_{\text{exp}}$) required to simulate time evolution of a system during a time interval $t > 0$, with error bounded by $\epsilon > 0$, is such that

  \begin{equation}
    N_{\text{exp}} \leq 2L5^{2k}(L\tau)^{1+1/2k}/\epsilon^{1/2k}
  \end{equation}

  Where $\tau = \left\Vert \hat{H} \right\Vert t$, $2k$ is the order of a Suzuki-Trotter iteration, and $\epsilon \leq 1 \leq 2L5^{2k}$. Notice that by choosing a sufficiently high order, a Suzuki-Trotter scheme can emulate time evolution with almost linear complexity in time.

  In most cases, simulation of a quantum system by a digital quantum computer requires mapping its Hilbert space to a $2^M$-dimensional Hilbert space. In that case, the number of qubits required for simulation would be $M$. Clearly, all quantum operators in the simulated system's space should be mapped to operators on an $M$-qubit space, which would then be approximated by a universal set of gates. In the following section, demonstrations of this approach to the study of the Hubbard Model and the Electronic structure problem are introduced.

\section{Time Simulation of Spin 1/2 Models}
\label{sec:hubbard}
  
  On section \ref{sec:qtsVcts}, a comparison between classical time simulation and quantum time simulation was introduced. In particular, a scheme for integrating Schrödinger's equation for a $n$-particle system was proposed. This section introduces examples of simulation of Hamiltonians such as

  \begin{equation}
    \hat{H} = \sum_{\langle i,j \rangle} J_{ij}^{(x)} \hat{X}_i \hat{X}_j + J_{ij}^{(Y)} \hat{Y}_i \hat{Y}_j + J_{ij}^{(Z)} \hat{Z}_i \hat{Z}_j + \sum_i h_i^{(X)} \hat{X}_i + h_i^{(Y)} \hat{Y}_i + h_i^{(Z)} \hat{Z}_i
    \label{eq:HeisenbergHamiltonian}
  \end{equation}

  Defined over an arbitrary spin graph, and considering nearest neighbor interaction. Most of them are related to work from Las Heras et. al. \cite{HubbardSimul,HubbardSimulLasHeras} and Y. Salathé \cite{HeisenbergSimulLasHeras}. In this section, the importance of simulating such systems is discussed. Furthermore, the examples considered here are further generalized in following chapters to simulate any system whose Hamiltonian is of the shape of eq. \ref{eq:HeisenbergHamiltonian}, and its limitations are exemplified.

  \subsection{Digital simulation of two-spin models}

  As a first example, simulation of two-spin models carried out by Y. Salathé et. al. \cite{HeisenbergSimulLasHeras} are presented. In their work, a superconducting chip with two trasmon qubits is used to simulate two-spin interaction described by the Hamiltonian

  \begin{equation}
    \hat{H}_{1,2}^{x,y} = \frac{J}{2} \bigg( \hat{X}_1 \hat{X}_2 + \hat{Y}_1 \hat{Y}_2 \bigg)
    \label{eq:fundamentalSlatheGate}
  \end{equation}

  This means that they where able to evolve the qubits' state during time $t$, using microwave pulses on the chip, with unitary dynamics governed by Hamiltonian \ref{eq:fundamentalSlatheGate}. It is easy to see that by performing single qubit rotations, it is possible to emulate more complicated dynamics, for instance, an isotropic XYZ interaction

  \begin{equation}
    \hat{H}_{1,2}^{x,y,z} = J \bigg( \hat{X}_1 \hat{X}_2 + \hat{Y}_1 \hat{Y}_2 + \hat{Z}_1 \hat{Z}_2 \bigg)
    \label{eq:XYZSalathe}
  \end{equation}

  An algorithm for simulating time dynamics under Hamiltonian \ref{eq:XYZSalathe} is presented on fig. \ref{fig:salathe-xyz}. The reported state fidelities of the evolution process are above $82\%$. It must be noted that, since all terms of the Hamiltonian commute, the evolution of the two qubit state is exact, and only limited by hardware. In addition to that, an algorithm for simulating the Ising model with transverse homogeneous magnetic field was proposed (see fig. \ref{fig:salathe-ising}).
  
  \begin{equation}
    \hat{H}_I = J \hat{X}_1 \hat{X}_2 + \frac{B}{2}\bigg(\hat{Z}_1 + \hat{Z}_2 \bigg)
    \label{eq:IsingSlathe}
  \end{equation}

  Notice that this Hamiltonian is composed of local parts that do not commute with one another (spin interaction and field interaction), thus an approximate evolution scheme is needed to simulate time evolution. In their work, Salathé et. al. \cite{HeisenbergSimulLasHeras} used a second order trotterization (see eq. \ref{eq:2ndOrderTrotter}). Furthermore, a noise model was proposed to take into account the fact that decoherence and gate errors limit the expected precision of a trotterization scheme. The results show that the main source of error is the infidelity of the two-qubit gate implementation of the XY interaction (eq. \ref{eq:fundamentalSlatheGate}).

  \subsection{Digital simulation of Hubbard models}

  As a second example, the work of las Heras et. al. is considered\cite{HubbardSimulLasHeras}. The authors simulated instances of the quintessential Hubbard Hamiltonian

  \begin{equation}
    \hat{H}_H = -V \sum_{\langle i,j\rangle} (\hat{b}_i^{\dagger}\hat{b}_j + \hat{b}_j^{\dagger}\hat{b}_i) + U \sum_i \hat{n}_{i\uparrow}\hat{n}_{i\downarrow}
    \label{eq:HubbardHamiltonian}
  \end{equation}

  Where $\hat{b}_i$ represent fermionic-mode annihilation operators (for spin-up or spin-down particles), and $\hat{n}_{i\uparrow},\hat{n}_{i\uparrow}$, fermionic-mode occupation number operators. This was done using the Jordan-Wigner mapping, which transforms fermionic operators to qubit operators. The rule is that each occupation mode is mapped directly to the state of a qubit, so that its state encodes the occupation number of the mode in the computational basis. Fermionic annihilation operators are mapped directly following the rule \cite{Mastersthesis}

  \begin{equation}
    \hat{b}_i = \hat{\sigma}_i^{+} \otimes \Bigg(\bigotimes_{k=0}^{N} \hat{Z}_k \Bigg)
    \label{eq:JWT}
  \end{equation}

  Where it is assumed that a linear indexing of the modes is used, even for different spin values. The authors considered models with up to four modes. In particular, they showed that the mapping of a two-mode Hubbard Hamiltonian like eq. \ref{eq:HubbardHamiltonian} leads to a qubit Hamiltonian

  \begin{equation}
    \hat{H}_H = \frac{V}{2}\big(\hat{X}_1 \hat{X}_2 + \hat{Y}_1 \hat{Y}_2\big) + \frac{U}{4} \big(\hat{Z}_1 \hat{Z}_2 + \hat{Z}_1 + \hat{Z}_2\big)
    \label{eq:TwoModeHubbard}
  \end{equation}

  The authors used a superconducting chip for simulating time dynamics governed by Hamiltonian \ref{eq:TwoModeHubbard}. The quantum algorithm is represented on figure \ref{fig:heras-hubbard}. It was used to simulate time evolution with an initial state
  
  $$
  \ket{\psi_0} = \frac{1}{\sqrt{2}}(\ket{0} + \ket{1}) \otimes \ket{0}
  $$

  With $V=U=1$, for a time $t=5$. The results showed that process errors in the implementation of a Trotter step leads to a linear decrease in state fidelity with the number of steps. This is an important factor to consider when implementing Suzuki-Trotter schemes for simulating time evolution. However, their simulations were able to capture the overall dynamics, obtaining fidelities near $90\%$ with a small number of Trotter steps. Three and four mode anisotropic models were simulated as well, obtaining similar results \cite{HubbardSimulLasHeras}.

  It has been shown by Reiner \cite{Mastersthesis} that a one dimensional Hubbard model (eq. \ref{eq:HubbardHamiltonian}) can be mapped to a qubit Hamiltonian of the type \ref{eq:HeisenbergHamiltonian}, by means of the Jordan Wigner mapping. Thus, by generalizing the quantum algorithms presented on these examples, it is possible to study correlation phenomena on metallic solids efficiently, in contrast to current classical simulation methods \cite{Mastersthesis,HubbardOriginal}. In following chapters,  general routines for simulating evolution under Hamiltonian \ref{eq:HeisenbergHamiltonian} are presented, and quantum time evolution is showcased as a tool for studying magnetic properties of solids. Moreover, a discussion of the influence of decoherence and errors in the implementation is carried out, thus presenting the limitations of Suzuki-Trotter schemes for simulating many-body systems.

  \begin{figure}
    \centering
    \begin{subfigure}[b]{1.0\textwidth}
        \centering
        \caption{}
        \begin{quantikz}
            \lstick[wires=2]{$\ket{\psi(0)}$} & \gate[wires=2]{\mathrm{e}^{-\mathrm{i}\hat{H}_{1,2}^{x,y}t}} & \gate{R_{\hat{x}, -\pi/2}} & \gate[wires=2]{\mathrm{e}^{-\mathrm{i}\hat{H}_{1,2}^{x,y}t}} & \gate{R_{\hat{x}, \pi/2}} & \gate{R_{\hat{y}, -\pi/2}} & \gate[wires=2]{\mathrm{e}^{-\mathrm{i}\hat{H}_{1,2}^{x,y}t}} &  \gate{R_{\hat{y}, \pi/2}} & \qw \rstick[wires=2]{$\ket{\psi(t)}$}\\
             & & \gate{R_{\hat{x}, -\pi/2}} & & \gate{R_{\hat{x}, \pi/2}} & \gate{R_{\hat{y}, -\pi/2}} & & \gate{R_{\hat{y}, \pi/2}} & \qw
        \end{quantikz}
        \label{fig:salathe-xyz}
    \end{subfigure}
    \vfill
    \begin{subfigure}[b]{1.0\textwidth}
        \centering
        \caption{}
        \begin{quantikz}
            \gate{R_{\hat{x}, \pi}} & \gate[wires=2]{\mathrm{e}^{-\mathrm{i}\hat{H}_{1,2}^{x,y}\frac{t}{N}}} & \gate{R_{\hat{x}, -\pi}} & \gate[wires=2]{\mathrm{e}^{-\mathrm{i}\hat{H}_{1,2}^{x,y}\frac{t}{N}}} &  \gate{R_{\hat{z}, \frac{t}{N}}} \\
            \qw & & \qw & & \gate{R_{\hat{z}, \frac{t}{N}}}
        \end{quantikz}
        \label{fig:salathe-ising}
    \end{subfigure}
    \begin{subfigure}[b]{1.0\textwidth}
        \centering
        \caption{}
        \begin{quantikz}
            \gate{R_{\hat{y}, \pi/2}} & \gate[wires=2]{\mathrm{e}^{-i\frac{\phi_{xx}}{2}\hat{Z}\otimes\hat{Z}}} & \gate{R_{\hat{y}, -\pi/2}} & \gate{R_{\hat{x}, -\pi/2}} & \gate[wires=2]{\mathrm{e}^{-i\frac{\phi_{yy}}{2}\hat{Z}\otimes\hat{Z}}} & \gate{R_{\hat{x}, \pi/2}} & \gate[wires=2]{\mathrm{e}^{-i\frac{\phi_{zz}}{2}\hat{Z}\otimes\hat{Z}}} & \gate{R_{\hat{z}, \phi_z}} \\
            \gate{R_{\hat{y}, \pi/2}} & & \gate{R_{\hat{y}, -\pi/2}} & \gate{R_{\hat{x}, -\pi/2}} & & \gate{R_{\hat{x}, \pi/2}} & & \gate{R_{\hat{z}, \phi_z}}
        \end{quantikz}
        \label{fig:heras-hubbard}
    \end{subfigure}
    \caption{(a) Digital quantum algorithm for simulating time dynamics of isotropic XYZ model according to \cite{HeisenbergSimulLasHeras}. Notice that commutation of $\hat{X}_1\hat{X}_2$, $\hat{Y}_1\hat{Y}_2$, and $\hat{Z}_1\hat{Z}_2$ implies that the resulting dynamics is exact up to hardware errors. (b) Trotter step for simulating of the Ising model with a transverse magnetic field. This step is repeated $N$ times to evolve over a time interval $t$ \cite{HeisenbergSimulLasHeras}. (c) Trotter step for simulating a two-mode Hubbard model. Here $\phi_{zz} = \phi_z = Ut/2N$ and $\phi_{xx} = \phi_{yy} = Ut/2N$ \cite{HubbardSimulLasHeras}.}
    \label{fig:salathe-includes}
\end{figure}

  

  

Quantum Computing has become a reality now that the field is entering the so called NISQ (Noisy Intermediate Quantum) era. The digital quantum circuit model is one of the most extended and discussed. It is based upon the concept of qubit, which is an abstract two-level quantum system. By profiting linear superposition and entanglement of many-qubit systems, quantum algorithms are thought to be more resource-efficient than standard classical algorithms in areas like machine learning and mathematical finance.The present work studies a more fundamental, yet quite versatile, application of quantum computation: simulation of quantum physical systems. This perspective was introduced in 1982 by Richard Feynman. He suggested that using a quantum system to simulate another can reduce the exponential overhead that occurs by incrementing the number of components. In particular, simulation of a Heisenberg Hamiltonian on an arbitrary graph is considered in the context of the quantum circuit model.

Even though this Hamiltonian is far simpler than arbitrary N-qubit systems Hamiltonians, it can be readily generalized to produce interesting behavior. For instance, it is well known that a Heisenberg chain exhibits a quantum phase transition at zero temperature. In the present work, a parametric Heisenberg Model is simulated, thus opening the possibility of extending the algorithm to the field of Quantum Machine Learning.

In particular, the recently proposed Variational Quantum Thermalizer (VQT) \cite{verdon2019quantum} is considered for producing thermal states of a magnetic system. A Quantum Neural Network (QNN) based upon time evolution of a parametric Heisenberg Hamiltonian is used to learn the thermal states of a magnetic system in a 2D system with non-square topology. In consequence, time evolution of an elementary Hamiltonian is used to study thermal properties of a quantum system, thus illustrating the potential of digital quantum computation in material science.

After a review of the main concepts of quantum computing (\autoref{chap:qc}), the fundamentals of quantum time simulation are revised on chapter \autoref{chap:qts}. On \autoref{chap:main}, the algorithm for simulating evolution of a Heisenberg spin system, with nearest neighbor interaction is presented. Simulations using Qiskit, as well as experimental results of execution on IBM Q devices are presented. Experimental and theoretical state fidelities produced by the algorithm are discussed in this chapter too. Finally, on \autoref{chap:vqt}, a quantum neural network is implemented that computes thermal states of a spin system using the Variational Quantum Thermalizer. This QNN is based upon the algorithm devised on \autoref{chap:main}.
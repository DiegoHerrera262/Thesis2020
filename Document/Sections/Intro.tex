Quantum Computing has become a reality now that the field is entering the so called NISQ (Noisy Intermediate Quantum) era. The digital quantum circuit model is one of the most extended and discussed. It is based upon the concept of qubit, which is an abstract two-level quantum system. By profiting linear superposition and entanglement of many-qubit systems, quantum algorithms are thought to be more resource-efficient than standard classical algorithms in areas like machine learning and mathematical finance. The present work studies a more fundamental, yet quite versatile, application of quantum computation: simulation of quantum physical systems. This perspective was introduced in 1982 by Richard Feynman \cite{Feynman1982}. He suggested that using a quantum system to simulate another can reduce the exponential overhead that occurs by incrementing the number of components. 

In particular, this work is concerned with the task of devising a quantum time simulator capable of performing evolution of parametric Heisenberg anisotropic models. A digital quantum algorithm for such tasks, based on Trotter decomposition, is devised using Qiskit SDK. At the moment of presentation of this dissertation, Las Heras et. al. \cite{HubbardSimulLasHeras} and Y. Salathé et. al. \cite{HeisenbergSimulLasHeras} have proposed simple digital quantum algorithms for simulating time evolution of small Heisenberg systems. The circuits employed by those groups, however, where specifically optimized for superconducting chips whose architecture is not publicly available. This work extends those insights and adapts them to the architecture available publicly by IBM Quantum through Qiskit SDK. Furthermore, effects of noise in Trotter-based simulation algorithms are discussed and simulated, as well as the capabilities of quantum time evolution as a subroutine for studying magnetic properties of solids.

Since 2021, Qiskit has extended their circuit library to include simulation of multi-qubit Hamiltonians using Suzuki-Trotter schemes. Most of those schemes are based upon a direct partition of the target hamiltonian in a Pauli-product basis. In the present work, a slightly more sophisticated approach, based on the nature of the specific Heisenberg model Hamiltonian is shown to produce lower circuit depth and a shorter execution time, for given accuracy.

This dissertation is structured in 4 chapters. After a review of the main concepts of quantum computing (\autoref{chap:qc}), the fundamentals of quantum time simulation are revised on \autoref{chap:qts}. Here, as an illustration, the circuits proposed by Las Heras et. al. \cite{HubbardSimulLasHeras} and Y. Salathé et. al. \cite{HeisenbergSimulLasHeras} are simulated using Qiskit SDK. On \autoref{chap:main}, these circuits are revisited and optimized in order to produce lower circuit depths, and potentially better state fidelity. Experimental and theoretical state fidelities produced by the algorithm are discussed in this chapter too. Moreover, the influence of noisy channels in the evolution schemes is discussed using models from Qiskit Ignis. Finally, on \autoref{chap:vqte}, due to the limitations of current quantum devices, a variational quantum time evolution algorithm is introduced and implemented.
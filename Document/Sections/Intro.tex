Quantum Computing has become a reality now that the field is entering the so called NISQ (Noisy Intermediate Quantum) era. The digital quantum circuit model is one of the most extended and discussed. It is based on the concept of qubit, which is an abstract two-level quantum system. By profiting linear superposition and entanglement of many-qubit systems, quantum algorithms are thought to be more resource-efficient than standard classical algorithms in areas like machine learning and mathematical finance. The present work studies a more fundamental, yet quite versatile, application of quantum computation: simulation of quantum physical systems. This perspective was introduced in 1982 by Richard Feynman \cite{Feynman1982}. He suggested that using a quantum system to simulate another can reduce the exponential overhead that occurs by incrementing the number of components. In fact, Lloyd \cite{LloydNature} has proved that this is the case, and Trotter integration schemes can be used to efficiently simulate quantum systems that can be modeled by local interactions on s number of component.

In particular, this work is concerned with the task of devising a quantum time simulator capable of performing evolution of parametric Heisenberg anisotropic models. A digital quantum algorithm for such tasks, based on Trotter decomposition, is devised using Qiskit SDK. At the moment of presentation of this dissertation, Las Heras et. al. \cite{HubbardSimulLasHeras, HubbardSimul} and Y. Salathé et. al. \cite{HeisenbergSimulLasHeras} have proposed simple digital quantum algorithms for simulating time evolution of small Heisenberg systems. The circuits employed by those groups, however, where specifically optimized for superconducting chips whose architecture is not publicly available.

Furthermore, collaborations with IBM Quantum research teams \cite{DuplicatedRXZPulse, RXZPulseEfficient, MajoranaSimulation} have produced novel techniques for implementing two-qubit gates using efficient microwave pulse schedules based on the cross resonance interaction. This control technique is also used to simulate time evolution, for instance, in the context of QAOA algorithms \cite{RXZPulseEfficient}. This work extends those insights and adapts them to the architecture available publicly by IBM Quantum through Qiskit SDK. Using commutation properties of local Hamiltonians, and direct pulse control via cross resonance gates, evolution of anisotropic Heisenberg Hamiltonians is simulated on quantum devices.

This dissertation is structured as follows. After a review of the elements of digital quantum time simulation, three simulation algorithms are discussed and compared in terms of probability density fidelity, state fidelity, and measurement of common observables. These correspond to direct trotterization using basis gates, trotterization using commutation properties, and trotterization with commutation and direct pulse control. A three-spin Heisenberg Hamiltonian is used as a benchmark model; however, simulation is readily extensible to more complex interactions. Since the three algorithms use the same integration scheme, a discussion of the inherent discretization error is performed using QASM simulators. Experimental results are obtained using \textit{IBMQ Jakarta} backend. Metrics on fidelity and measured observables mentioned before are finally compared and the utility of cross resonance pulses and symmetries of local Hamiltonians is made explicit.
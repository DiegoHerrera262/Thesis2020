This chapter introduces the elements of quantum computing. First, the qubit as a two-level system and a quantum register as a \textit{multi-qubit} system are presented. Then, unitary operations on a quantum register's Hilbert space are introduced as the quantum analogue of digital logic gates. This leads to the question about how to represent any unitary operator on a quantum register using quantum gates. Thereby introducing the concept of universal basis gates sets. Finally, a digression about the capabilities of quantum computing for the general theory of computations is presented.

\section{Quantum circuit model elements}

  Nowadays, it is familiar to understand information processing tasks in terms of \textit{bits}. A bit is a binary variable that can be said to have either of two states on a set (commonly denoted by $0$, $1$). As a result, a bit is nothing more than a variable $s$ whose value is in the set $\{0,1\}$. In modern computers, such an entity can be encoded physically by means of electric current or voltage. Most operations that can act upon a bit are mappings from one of the possible state values to the other. These corresponds to two fundamental operations: the NOT gate, which flips the bit value, or the IDENTITY gate, whose action is actually inaction.

  In contrast, a quantum bit or \textit{qubit}, corresponds physically to a two-level quantum system, whose Hilbert space can be spanned by a \textit{computational basis set} $\{\ket{0}, \ket{1}\}$. It is reasonable to assume that this basis set is orthonormal

  \begin{equation}
    \braket{s}{s'} = \delta_{ss'} \text{ for } s, s' \in \{0,1\}
    \label{eq:comp-basis}
  \end{equation}

  The canonical representation of a qubit is as a vector in the \textit{Bloch sphere}. This representation maps a generic qubit superposition state

  \begin{align}
    \cos(\frac{\theta}{2}) \ket{0} &  + \mathrm{e}^{\mathrm{i}\phi}\sin(\frac{\theta}{2})\ket{1} \\
    & \theta, \phi \in \mathcal{R}
    \label{eq:qubit-state}
  \end{align}

  To a vector on the unitary 3-sphere

  \begin{equation}
    \hat{\mathbf{n}} = \sin(\theta) \cos(\phi) \hat{\mathbf{x}}  + \sin(\theta) \sin(\phi)\hat{\mathbf{y}} + \cos(\theta) \hat{\mathbf{z}}
    \label{eq:bloch-vector}
  \end{equation}

  In addition to the computational basis, other important states are the \textit{sign basis}

  \begin{gather}
    \ket{+} = \frac{1}{\sqrt{2}}\ket{0} + \frac{1}{\sqrt{2}}\ket{1} \\
    \ket{-} = \frac{1}{\sqrt{2}}\ket{0} - \frac{1}{\sqrt{2}}\ket{1}
    \label{eq:sign-basis}
  \end{gather}

  And the \textit{y basis}

  \begin{gather}
    \ket{+\mathrm{i}} = \frac{1}{\sqrt{2}}\ket{0} + \mathrm{i}\frac{1}{\sqrt{2}}\ket{1} \\
    \ket{-\mathrm{i}} = \frac{1}{\sqrt{2}}\ket{0} - \mathrm{i}\frac{1}{\sqrt{2}}\ket{1}
    \label{eq:phase-basis}
  \end{gather}

  Those states actually lie in the poles of the Bloch sphere, and each pair constitutes a basis for a qubit's Hilbert space that has interesting properties regarding measurement.

  In contrast to classical bits, a quantum bit can be transformed by an infinite number of operations, corresponding to all possible operators defined on its Hilbert space. Of those, two are of huge importance in quantum computing: unitary operators and measurement operators. Unitary operations are used mainly to process quantum information and perform a computation. These are commonly known as \textit{quantum gates}. The fundamental quantum gates are the Pauli operators

  \begin{gather}
    \hat{X} = \ketbra{+}{+} - \ketbra{-}{-} \\
    \hat{Y} = \ketbra{+\mathrm{i}}{+\mathrm{i}} - \ketbra{-\mathrm{i}}{-\mathrm{i}} \\
    \hat{Z} = \ketbra{0}{0} - \ketbra{1}{1}
    \label{eq:pauli-ops}
  \end{gather}
  
  
  For a single qubit, it is known that any unitary operation can be represented as rotation operator of the form

  \begin{gather}
    \hat{R}_{\hat{n}, \theta} = \cos(\frac{\theta}{2}) - \mathrm{i}\sin(\frac{\theta}{2}) \hat{n} \vdot \sigma \\
    \hat{n} \vdot \sigma = n_x \hat{X} + n_y \hat{Y} + n_z \hat{Z} \\
    ||\hat{n}||^2 = 1
  \end{gather}

  In the Bloch sphere representation, quantum gates, therefore, correspond to rotations of the quantum state's associated vector (eq. \ref{eq:bloch-vector}). A quite important rotation is the so called \textit{Hadamard gate}, whose representation in the computational basis is
  
  \begin{equation}
    \hat{H} = \frac{1}{\sqrt{2}}\begin{bmatrix}
      1 & 1 \\
      1 & -1
    \end{bmatrix}
    \label{eq:hadamard-gate}
  \end{equation}
  
  
  These rotations allow computations, but the actual process of reading the outcome of an algorithm implies measurement. In current digital quantum computing implementations, measurement operators are projectors onto the computational basis

  \begin{gather}
    \hat{P}_0 = \ketbra{0}{0} \\
    \hat{P}_1 = \ketbra{1}{1} 
    \label{eq:measurement-ops}
  \end{gather}

  These operators allow measurement of the expected value of $\hat{Z}$ operator. By performing rotations to the sign and y basis, it is possible to measure expected values of $\hat{X}$ and $\hat{Y}$, respectively. Since Pauli operators span the space of qubit operators, any qubit observable can be measured by applying suitable rotations and forming linear combinations of expected values of Pauli operators.

  \subsection{Quantum registers and multi-qubit gates}

    In general, more than one qubit is needed to perform meaningful computations. A system of several qubits is called \textit{quantum register}. A quantum register's Hilbert space is nothing more than the tensor product space of each of its constituent qubits. Thus, a $N$-qubit register has a $2^N$-dimensional Hilbert space. A basis for this space is built by all possible tensor products of computational basis states for each qubit. This would be the \textit{computational basis} of the register. A member of this set may be denoted by

    \begin{gather}
      \ket{s_{N-1}s_{N-2} \cdots s_0} = \ket{s_{N-1}} \otimes \ket{s_{N-2}}\otimes \cdots \otimes \ket{s_{0}} \\
      s_{N-1}, s_{N-2}, \ldots, s_0 \in \{0,1\}
      \label{eq:multi-qubit-basis}
    \end{gather}

    There may be other basis of interest in quantum computing. For instance, with $N=2$, the so called \textit{Bell basis},

    \begin{gather}
      \ket{\Phi^{+}} = \frac{1}{\sqrt{2}}(\ket{00} + \ket{11})\\
      \ket{\Phi^{-}} = \frac{1}{\sqrt{2}}(\ket{00} - \ket{11})\\
      \ket{\Psi^{+}} = \frac{1}{\sqrt{2}}(\ket{10} + \ket{01})\\
      \ket{\Psi^{=}} = \frac{1}{\sqrt{2}}(\ket{10} - \ket{01})
      \label{eq:BellBasis}
    \end{gather}

    is of capital importance in quantum information theory and quantum communications. It will also prove to be important in this work. Much like single-qubit gates, register gates or multi-qubit gates correspond to unitary operators defined on the register's Hilbert space. An important multi-qubit gate, defined by its action on two qubits is CNOT

    \begin{gather}
      \text{CNOT}\ket{\psi}_t \otimes \ket{1}_c = \big(\hat{X}\ket{\psi}_t\big) \otimes \ket{0}_c\\
      \text{CNOT}\ket{\psi}_t \otimes \ket{0}_c = \ket{\psi}_t \otimes \ket{0}_c
      \label{eq:cnot-gate}
    \end{gather}

    Where the subscript $t$ denotes the \textit{target} qubit, and the subscript $c$, the control qubit. Hence, CNOT corresponds to a controlled-$\hat{X}$ operation. This gate can entangle separable two-qubit states. This is of vital importance for universal quantum computing, and can be readily seen from the definition.

  \subsection{Circuit representation}

    This model of computation can be represented graphically by a \textit{circuit}, in which every wire represents a quantum bit, and a gate corresponds to a unitary operator acting on the register's Hilbert space (possibly a Hilbert subspace). For instance, an algorithm for producing a Bell basis state can be represented as in fig. \ref{fig:bell-pair-circuit}

    \begin{figure}[H]
    \centering
    \begin{quantikz}
        \lstick{$\ket{0}$} & \gate{H} & \ctrl{1} & \qw \\
        \lstick{$\ket{0}$} & \qw      & \targ{}  & \qw
    \end{quantikz}
    \caption{Simple algorithm for producing Bell state $\ket{\Phi^{+}}$}
    \label{fig:bell-pair-circuit}
\end{figure}

    In this circuit representation, the algorithm depicted consists on an application of a hadamard operator to a control qubit, followed by a CNOT operation, to a two qubit register on state $\ket{00}$. In general, quantum gates are represented by boxes, labeled properly by the operation that represent. These boxes cover the subspace over which the corresponding quantum operator acts. For instance, in the case of the algorithm depicted on fig. \ref{fig:bell-pair-circuit}, the Hadamard gate acts on a single-qubit subspace, where the CNOT gate acts on the whole two-qubit register's space. Operations are executed from left to right in a quantum algorithm.

\section{Universal quantum computing}
  
  As was stated before, all quantum gates correspond to unitary operations (or measurements), acting on a quantum register. It is desirable to find a set of elementary quantum gates, that act on a small number of qubits, that can generate all possible unitary operations on an $N$-qubit register. This is the question of universal quantum computing. It can be shown that CNOT and the set of single-qubit unitary operations are enough for producing all possible $N$-qubit register gates \cite{Nielsen}. For instance, it is possible to perform the three-qubit \textit{Controlled CNOT} operation

  \begin{gather}
    \text{CCNOT} \ket{\psi} \otimes \ket{s_t} \otimes \ket{s'_t} = \big(\hat{X}^{s_t s'_t} \ket{\psi}\big) \otimes \ket{s_t} \otimes \ket{s'_t} \\
    s_t, s'_t \in \{0,1\}
    \label{eq:toffoli-definition}
  \end{gather}

  by following algorithm on figure \ref{fig:toffoli-universal}. In general, however, performing arbitrary single-qubit rotations by hardware operations is impractical. This would require a huge control on the physical qubits that is not yet available for all possible architectures \cite{Nielsen}. As a result, most quantum processors available today use a limited set of single-qubit rotations for \textit{approximating} arbitrary rotations.

  \begin{figure}
    \centering
    \begin{quantikz}
        \qw & \ctrl{2} & \qw \\
        \qw & \ctrl{1} & \qw \\
        \qw & \targ{}  & \qw
    \end{quantikz}
    =\begin{quantikz}
        \qw & \qw      & \qw      & \qw              & \ctrl{2} & \qw      & \qw & \qw & \ctrl{2} & \qw & \ctrl{1} & \qw & \ctrl{1} & \gate{T} & \qw \\
        \qw & \qw      & \ctrl{1} & \qw              & \qw      & \qw      & \ctrl{1} & \qw & \qw & \gate{T^\dagger} & \targ{} & \gate{T^\dagger} & \targ{} & \gate{S} & \qw \\
        \qw & \gate{H} & \targ{}  & \gate{T^\dagger} & \targ{}  & \gate{T} & \targ{} & \gate{T^\dagger} & \targ{} & \gate{T} & \gate{H} & \qw & \qw & \qw & \qw
    \end{quantikz}
    \caption{Representation of CCNOT (L. H. S.) gate in terms of single-qubit rotations and CNOT (R. H. S.), according to \cite{Nielsen}. Here $\hat{S} = \sqrt{\hat{Z}}$ and $\hat{T} = \sqrt{\hat{S}}$.}
    \label{fig:toffoli-universal}
\end{figure}

  A commonly used universal set is $\{\hat{H}, \hat{S} = \sqrt{\hat{Z}}, \hat{T} = \sqrt{\hat{S}}\}$. With this set, it is possible to approximate any single qubit rotation to an arbitrary precision, but not exactly, using a finite number of operations. In contrast, IBM Quantum devices use a basis set $\{\hat{S}_x = \sqrt{\hat{X}}, \hat{X}, \hat{R}_{z,\phi} \}$, which can reproduce any single-qubit rotation exactly. As an example, the Hadamard gate can be decomposed as follows

  \begin{equation}
    \hat{H} = \hat{R}_{z,\pi/2} \hat{S}_x \hat{R}_{z,\pi/2}
    \label{eq:hadamard-decomp}
  \end{equation}

  As may be inferred from the two examples above, emulating arbitrary $N$-qubit-register operators can cause some computational overhead, that is an increase in the number of elementary operations required to reproduce a quantum computation. Usually, the basis set depends on the architecture of a quantum processor, and thus, due to current limitations of available hardware, it is important to design a quantum algorithm so that the operations involved can be optimally represented by the universal set associated to the device on which it is expected to be implemented.

\section{Quantum computational advantage}
  
  It is expected that quantum computing paradigms challenge the strong Church-Turing thesis, which claims that the ultimate reference for computational complexity is the probabilistic Turing Machine Model \cite{Nielsen, Strini}. In principle, a quantum processor could solve problems that are thought to be hard for probabilistic Turing Machines. Such problems include simulation of chemical systems \cite{QuantumChem1, QuantumChem2}, and multi-particle quantum systems in general \cite{Nielsen, LloydFermSim, BerryErrorBounds}. Regarding the quantum circuit model of computation, this usually means that the total gate count of a circuit that implements an algorithm that solves a hard problem is bounded asymptotically by a polynomial function in the size of the input \cite{Nielsen, Strini}. In general, however, a more accurate measurement of the complexity of a quantum algorithm is the \textit{circuit depth}, which measures the maximum number of operations a qubit has to undergo to complete a computation. One of the goals of this work is to show that it is possible to simulate arbitrarily large spin systems efficiently using quantum digital algorithms, something impossible with the standard methods of computational quantum physics.
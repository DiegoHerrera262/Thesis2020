\begin{figure}
    \centering
    \begin{quantikz}
        & \gate[wires=2]{\mathrm{e}^{-\mathrm{i}\hat{H}_{0,1}\Delta t}} & \qw & \gate{\mathrm{e}^{-\mathrm{i}\hat{H}_{0}\Delta t}} & \qw \\
        & \qw & \gate[wires=2]{\mathrm{e}^{-\mathrm{i}\hat{H}_{1,2}\Delta t}} & \gate{\mathrm{e}^{-\mathrm{i}\hat{H}_{1}\Delta t}} & \qw \\
        & \gate[wires=2]{\mathrm{e}^{-\mathrm{i}\hat{H}_{2,3}\Delta t}} & \qw & \gate{\mathrm{e}^{-\mathrm{i}\hat{H}_{2}\Delta t}} & \qw \\
        & \qw & \gate[wires=2]{\mathrm{e}^{-\mathrm{i}\hat{H}_{3,4}\Delta t}} & \gate{\mathrm{e}^{-\mathrm{i}\hat{H}_{1}\Delta t}} & \qw \\
        & \qw & \qw  &      \gate{\mathrm{e}^{-\mathrm{i}\hat{H}_{4}\Delta t}} & \qw
    \end{quantikz}
    \caption{Trotter step for simulating a spin chain. Notice that spin-spin terms that do not share a graph point can be evolved in parallel. A greedy algorithm can be used to determine a possible, if not optimal, scheme for simulating spin-spin interactions on parallel.}
    \label{fig:spinChainCircuit}
\end{figure}
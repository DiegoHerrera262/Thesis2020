\begin{figure}
    \centering
    \begin{subfigure}[b]{1.0\textwidth}
        \centering
        \caption{}
        \begin{quantikz}
            & \ctrl{1} & \gate{H}\slice{(1)} & \gate{\hat{R}_{\hat{z}, \phi_{xx}}}\slice{(2)} & \ctrl{1}  & \qw                                 \slice{(3)} & \ctrl{1} & \gate{H} & \ctrl{1} & \qw \\
            & \targ{}  & \qw                    & \gate{\hat{R}_{\hat{z}, \phi_{zz}}}         & \targ{}   & \gate{\hat{R}_{\hat{z}, \phi_{yy}}}             & \targ{}  & \qw      & \targ{}  & \qw
        \end{quantikz}
    \end{subfigure}
    \begin{subfigure}[b]{1.0\textwidth}
        \centering
        \caption{}
        \begin{quantikz}
            & \ctrl{1} & \gate{H}\slice{(1)} & \gate{\hat{R}_{\hat{z}, \phi_{xx}}}\slice{(2)} & \ctrl{1}  & \qw                                 \slice{(3)} & \gate{H} & \ctrl{1} & \gate{\hat{R}_{\hat{x}, \pi/2}} & \qw \\
            & \targ{}  & \qw                    & \gate{\hat{R}_{\hat{z}, \phi_{zz}}}         & \targ{}   & \gate{\hat{R}_{\hat{z}, \phi_{yy}}}             & \qw      & \targ{}  & \gate{\hat{R}_{\hat{x}, -\pi/2}} & \qw
        \end{quantikz}
    \end{subfigure}
    \caption{(a) Implementation of spin-spin interaction using abelian groups. The stages are devided according to the reasoning on the main text. The last stage is de direct approach to rearranging the Bell basis (see eq. \ref{eq:SpinSpinPhases}). (b) Simulation of spin-spin interaction with three CNOT gates. The last stage is simplified by considering some properties of Bell states. This dissertation started with the circuit on (a). The trick for reducing the implementation to three CNOT gates was developed on \cite{BellUniversalCartan}. The trick of leveraging the commutation properties of the Hamiltonian was thought of independently by the author of the present work and those on the mentioned reference.}
    \label{fig:abelianTrotterSpinSpin}
\end{figure}